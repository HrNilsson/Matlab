\documentclass[MiniProjectMain]{subfiles}
\begin{document}

\chapter{Theory}

\section{Cyclic syndrome decoding}
The syndrome of a cyclic code can be calculated efficiently using a sequential division circuit, based on the generator polynomial.
This is done by shifting the entire received vector into the division circuit.

The syndrome of the received vector is now readable in the delay line of the division circuit.
If the circuit is then shifted further, with no input but except any correction bits, the next syndrome is readable in the delay line of the division circuit.

This is employed in the Meggitt decoder to continuously check for errors in the received vector.

\section{Meggitt decoding}

\subsection{The Meggitt decoding circuit}
In this the circuit of the Meggitt decoder is presented.

% Figur

The Meggitt decoder consist of:
\begin{itemize}
\item
A receive buffer \textit{r(x)}

\item
A Buffer register

\item
A division circuit with a Syndrome register

\item
An Error Pattern detection circuit

\subitem This circuit normally consist of combinational logic for single error detection, and table lookup in multi error detection.

\item
A Code register containing the corrected vector.

\end{itemize}

\subsection{The Meggitt decoding algorithm}













\end{document}